We are living in an ever more connected world, where data recording the
interactions between people, software systems, and the physical world is
becoming increasingly prevalent. This data often takes the form of a temporally
evolving graph, where entities are the vertices and the interactions between
them are the edges. We call such graphs interaction graphs. Various application
domains, including telecommunciations and social media, depend on analytics
performed on interaction graphs. The ability to efficiently support historical
analysis over interaction graphs require effective solutions for the problem of
data layout on disk. 

This paper presents an adaptive disk layout called the railway layout for
optimizing disk block storage for interaction graphs. The key idea is to divide
blocks into one or more sub-blocks, where each sub- block contains a subset of
the attributes, but the entire graph structure is replicated within each
sub-block.  We introduce optimal ILP formulations for partitioning disk blocks
into sub-blocks with overlapping and non-overlapping attributes. Additionally,
wepresent greedy heuristic approaches that can scale better compared to the ILP
alternatives, yet achieve close to optimal query I/O. To demonstrate the
benefits of the railway layout, ee provide an extensive experimental study
comparing our approach to a few baseline alternatives.